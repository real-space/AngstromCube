\documentclass[oribibl]{llncs}

\usepackage{units}
\usepackage{psfrag} %% psfrac does not work with pdflatex
\usepackage{amssymb}
\usepackage{amsmath}
\usepackage{booktabs}
\usepackage{microtype}
\usepackage{subfigure}
\usepackage{todonotes} %% add [disable] to disable
\usepackage{transparent}
\usepackage{pgfplots}

\usepackage{acronym} %% for abbreviations
\acrodef{dft}[DFT]{density functional theory}
\acrodef{sho}[SHO]{spherical harmonic oscillator}
\acrodef{dos}[DoS]{density of states}
\acrodef{ho}[HO]{harmonic oscillator}
\acrodef{planewave}[PW]{plane wave}
\acrodef{gridpoint}[GP]{grid point}
\acrodef{paw}[PAW]{Projector Augmented Wave}


\setlength{\tabcolsep}{6pt}

\newcommand{\um}[1]{_{\mathrm{#1}}}
\newcommand{\ttt}[1]{\texttt{#1}}
\newcommand{\lmax}{\ell_{\mathrm{max}}}
\newcommand{\ellm}{L}
\newcommand{\nrn}{n_{\mathrm{r}}}
\newcommand{\ket}[1]{\left| #1 \right\rangle}
\newcommand{\bra}[1]{\left\langle #1 \right|}
\newcommand{\braket}[2]{\left\langle \left. #1 \right| #2 \right\rangle}
\newcommand{\braketop}[3]{\left\langle \left. #1 \right| #2 \left| #3 \right. \right\rangle}
\newcommand{\codename}{A4cube}

\begin{document}
\pagestyle{plain}

\title       {\codename{}}
\titlerunning{\codename{}}

\author{%
  Paul F.~Baumeister\inst{1} % \and %
}

\institute{%
  J\"{u}lich Supercomputing Centre, Forschungszentrum J\"{u}lich, 52425 J\"{u}lich, Germany
%   \and Institute for Advanced Simulation, Forschungszentrum J\"{u}lich, 52425 J\"{u}lich, Germany
}

\maketitle

\begin{figure*}
	\centering
	\includegraphics[width=3cm]{fig/aa3_logo_bold_no_helper_lines} %%
\end{figure*}

% ==============================================================================
\begin{abstract}
This is a user manual for everyone who is only interested in using
this software.
As far as possible, all aspects of programming are omitted here.
However, some aspects related to the algorithms used will be touched.
\end{abstract}
% ==============================================================================


\newpage
% ==============================================================================
\section{Introduction} \label{sec:intro}
% ==============================================================================
%
Real-space grid based methods for \ac{dft}
have proven to yield good parallelizability and the same level of
accuracy as \ac{planewave} basis sets.
% The latter aspect is in particular true for results converged
% in terms of the number of \acp{planewave} compared to those converged
% w.r.t.~the grid spacing.
% However, real-space grid methods cannot be operated at very coarse
% grid spacings due to unphysical interferences between the position
% of atoms relative to the position of grid points.
% This leads to relatively high cost prefactors (for time and space)
% of the real-space methods compared to \ac{planewave} basis sets
% when we want to do a fast but less accurate calcuation.
% Furthermore, the iterative solver schemes applied in real-space
% grid formalism often deteriorate in terms of their convergence
% velocity when the number of grid points increases.
% This is due to the limited width of the stencil compared to the 
% global scale of the non-local solutions.
% Therefore, convergence acceleration is crutial here.
% In many situations, switching to a \ac{planewave} representation
% is a viable option for an efficient preconditioner as
% \acp{planewave} contain the full non-locality.
% Then again, \acp{planewave} destroy the parallelizability
% to some extend.
% The approach investigated in this project is a small basis of atom-centered
% localized orbitals.
% This basis is not meant to produce as accurate results as \acp{gridpoint} or \acp{planewave}
% or to feature a advantageous convergence w.r.t.~costs
% but to be simple and cheap in its construction (low number of control parameters)
% and to produce approximately right physics already at small basis sets.
% If in addition we can define an efficient transformation between
% representations in
% the small localized basis set and grid-based representations
% this basis can be used to accelerate
% the congerence of a real-space grid-based method.

% ==============================================================================
\section{Spherical Harmonic Oscillator} \label{sec:sho}
% ==============================================================================
%
The quantum-mechanical \ac{ho} has the Hamiltonian
\begin{equation}
  \hat H^{[1D]}_{\sigma} = -\frac{\partial_x^2}{2} + \frac{x^2}{2 \sigma^4} \text{.}
  \label{eqn:HO-Hamiltonian}
\end{equation}
Hartree atomic units are used throughout this document.
Here, $\sigma$ is a lengthscale parameter that also fixes the scale of the eigenenergies
\begin{equation}
  E^{[1D]}_{n}(\sigma) = \sigma^{-2} \left( n + \frac 12 \right) \text{.}
  \label{eqn:HO-eigenenergy}
\end{equation}
The \ac{ho} eigenfunctions are
\begin{equation}
  \psi^{[1D]}_{n}(x) = H_n(x/\sigma) \cdot \exp\left( -\frac{x^2}{2 \sigma^2} \right) 
  \label{eqn:HO-eigenfunction}
\end{equation}
with the Hermite polynomials $H_n$.
\todo[inline]{add normalization constants}

The eigenfunctions of the quantum-mechanical 
three-dimensional isotropic harmonic oscillator 
- in the following we will refer to it as \ac{sho} -
can be written as Cartesian product of three
\ac{ho} eigenfunctions
\begin{equation}
  \hat H^{[3D]}_{\sigma} = -\frac{\vec \nabla^2}{2} + \frac{\vec r^2}{2 \sigma^{4}} 
  \label{eqn:SHO-Hamiltonian}
\end{equation}
has the solutions
\begin{equation}
  \psi^{[3D]}_{n_x n_y n_z}(x,y,z) = \psi^{[1D]}_{n_x}(x/\sigma) 
                               \cdot \psi^{[1D]}_{n_y}(y/\sigma) 
                               \cdot \psi^{[1D]}_{n_z}(z/\sigma) 
  \label{eqn:SHO-eigenfunction}
\end{equation}
and the eigenenergies
\begin{equation}
  E^{[3D]}_{n_x n_y n_z}(\sigma) = \sigma^{-2} \left( n_x + n_y + n_z + \frac 32 \right) 
  \label{eqn:SHO-eigenenergy}
\end{equation}

Furthermore, the \ac{sho} can be solved in spherical coordinates 
exploiting its spherical symmetry.
This leads to the quantum numbers $\nrn, \ell$ and $m$
and the eigenstates
\begin{equation}
  \psi_{\nrn \ell m}(r,\vartheta,\varphi) = R_{\nrn \ell}(r/\sigma) 
                               \cdot Y_{\ell m}(\vartheta,\varphi)
  \label{eqn:SHO-eigenfunction-radial}
\end{equation}
with the radial function
\begin{equation}
  R_{\nrn \ell}(r) = r^\ell \cdot L^{(\ell + \frac 12)}_{\nrn}(r^2) \cdot \exp(-\frac{r^2}2)
  \label{eqn:SHO-radial-eigenfunction}
\end{equation}
where $L_n^{(\alpha)}$ stands for the associated Laguerre polynomials.
\todo[inline]{add normalization constants}

In the code, spherical harmonics $Y_{\ell m}$ usually appear in their representation
$X_{\ell \mu}$ where each $\mu$ stands for a real-valued linear combination of $m$ and -$m$.
\todo[inline]{cite Homeier}

\begin{figure}
	%%% how to generate this plot: 
	%%% ./aa3 -t hermite_polynomial. +verbosity=10
  \begin{minipage}[c]{.990\textwidth}
	\includegraphics[width=\textwidth]{fig/hermite_gauss_functions} %%
  \end{minipage}\hfill
  \begin{minipage}[c]{.009\textwidth}
  \end{minipage}
  \label{fig:hermite_gauss_function}
  \caption{
  Hermite-Gauss functions up to $n\um{max} = 5$.
  }
\end{figure}

% ==============================================================================
\bibliographystyle{splncs03} \bibliography{literature}
% ==============================================================================

\end{document}
